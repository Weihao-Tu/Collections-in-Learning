% !TEX encoding = UTF-8 Unicode
% !Mode:: "TeX:UTF-8"
\documentclass{resume}
\usepackage{zh_CN-Adobefonts_external} % Simplified Chinese Support using external fonts (./fonts/zh_CN-Adobe/) %\usepackage{zh_CN-Adobefonts_internal} % Simplified Chinese Support using system fonts
\usepackage{linespacing_fix} % disable extra space before next section
\usepackage{cite}
\begin{document}
\pagenumbering{gobble} % suppress displaying page number
\name{屠为浩}
\contactInfo{153-6579-7776}{tuweihao510@outlook.com}{求职意向:助理硬件工程师}{}
\section{个人基本情况}
\par {籍贯:江苏盐城} \par {出生年月:1998.6.29} \par {随时入职} \par {接受大小周工作时间,接受加班}

\section{教育背景}
\datedsubsection{\textbf{淮阴工学院},计算机科学与技术,\textit{工学学士}}{2018.9 - 2022.6} 
\ \textbf{英语四级},GPA(2.95/4.5),数据结构(90),大学物理实验(良好),电工电子实习1(优秀),网络工程技能训练(良好)
\datedsubsection{\textbf{复旦大学},电子信息,\textit{工学硕士}}{2024.9 - 2026.6} 
\ \textbf{英语六级}

\section{技术能力}
\begin{itemize}[parsep=0.2ex]
 \item \textbf{了解编程语言}: c,c++, Python, SQL, Shell
 \item \textbf{有嵌入式设计和调试,电路板焊接,硬件测试经验}
 \item \textbf{了解AD软件的使用}
  \item \textbf{办公技能}:office/ Latex
 \item \textbf{专业软件}:matlab/cfd
\end{itemize}

\section{项目经历}
\datedsubsection{\textbf{xxx温湿度自动监控系统}\qquad 校级大创项目}{2020.6-2020.11}
\begin{itemize}
\item \textbf{项目描述}:设计xxx的温湿度报警方案,该方案包括单片机最小系统、温湿度传感器检测模块DHT11、按键模块、报警模块、降温控制模块、加湿控制模块和电源部分
\item\textbf{工作内容}:负责主控模块电路原理图的设计,以及编写系统主程序代码,液晶器显示模块代码,和温湿度采集模块代码。
\end{itemize}

\section{工作经历}
\datedsubsection{\textbf{盐城市科博电子仪器有限公司}\qquad  环保调试工程师}{2022.6-2022.10}
%\begin{itemize}
%\item \textbf{项目描述}:基于单片机的xxx中的时控处理器。在硬件层面,使用数码管来进行显示。软件方面采用汇编语言编程。该时控处理器可实现采样倒计时设定。
%\item\textbf{工作内容}:编写时控处理器的主程序实现对时间的设置和修改,以及相应的中断程序。然后采用内部时钟的驱动方式来驱动单片机工作,利用发光二极管来表示秒钟,其他的显示电路部分由一个数码管构成,然后设计该处理器的报警电路和按键电路。在电源设计部分,采用合适的变压器和二极管组成桥式全波整流电路,然后接上电容和稳压器。
%\end{itemize}
\begin{itemize}
\item \textbf{项目描述}:负责对环保检测设备调试
\item\textbf{工作内容}:负责环境检测仪器调试售后工作,指导客户操作仪器。
\end{itemize}
\datedsubsection{\textbf{苏州云视达机器人科技有限公司}\qquad  视觉调试工程师}{2022.11-2023.3}
\begin{itemize}
\item \textbf{项目描述}:配合机器人相机,对产线上的产品进行三维重建。
\item\textbf{工作内容}:创建charuco板,然后进行相机标定,在目标区域初始化点云,建立世界坐标系, 并利用solvePnP解出相机位姿,阈值分割提取目标,通过相机位姿和目标像素重投影筛选出目标三维点并上色,最后通过viz显示出3D重建的效果。
\end{itemize}

\section{竞赛获奖}
\begin{itemize}[parsep=0.2ex]
\item 蓝桥杯编程竞赛c\c++组\textbf{江苏省二等奖},2020 年 9 月
\item 江苏省高数竞赛\textbf{江苏省三等奖},2021 年 9 月
\end{itemize}

\section{社区参与/实践其他}
\begin{itemize}[parsep=0.2ex]
\item 编程竞赛社团组织成员,\textbf{负责c语言讲解 }

\end{itemize}

\end{document}
